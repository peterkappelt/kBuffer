\hypertarget{index_intro}{}\section{Introduction}\label{index_intro}
k\+Buffer is a universal library for a ring-\/ / circular buffer. \hypertarget{index_function}{}\section{Functions and Datatypes}\label{index_function}
\hyperlink{structbuffer__t}{buffer\+\_\+t}~\newline
 \hyperlink{k_buffer_8h_a7a0bf550b7bd49d85172e409c0034fe6}{buffer\+Status\+\_\+t}~\newline
 ~\newline
 \hyperlink{k_buffer_8c_aec18d6ea571b1326dbeb7ca15f4969c0}{buffer\+Init()}~\newline
 ~\newline
 \hyperlink{k_buffer_8c_a9d6410a89adf65a3ef12340ecb9bbd55}{buffer\+Write()}~\newline
 \hyperlink{k_buffer_8c_a9b80be9033ccd6b5a101f811520ab4cc}{buffer\+Read()}~\newline
 \hyperlink{k_buffer_8c_a5ecef1fd460ed9635269abce02be866f}{buffer\+Peek()}~\newline
 ~\newline
 \hyperlink{k_buffer_8c_ac806d926fd21729feb18f8e7738e76b4}{buffer\+Fill()} ~\newline
 \hyperlink{k_buffer_8c_ac69b8a12a33d0cf0a5dab8feb4f7b020}{buffer\+Is\+Full()}~\newline
 \hyperlink{k_buffer_8c_a5c599b9386c73ccd7b5eeb25f6cca38e}{buffer\+Is\+Empty()}~\newline
 ~\newline
 \hyperlink{k_buffer_8c_a8508583be1e356a243b0ce67254c516e}{buffer\+Write\+To\+Index()}~\newline
 \hyperlink{k_buffer_8c_aa0d7e2a4b6fd3da2822d7f968be74e5c}{buffer\+Read\+From\+Index()}~\newline
 ~\newline
 \hyperlink{k_buffer_8c_afa8bb2b701cd9b7f871c12e0fabd66e1}{buffer\+Available()}~\newline
 ~\newline
 \hyperlink{k_buffer_8c_a1389f5c08210e077301c35bc3b43f681}{buffer\+Mean()}~\newline
 \hyperlink{k_buffer_8c_a1da694b34c0a52809c923d2d149d1348}{buffer\+Mean\+R\+M\+S()}~\newline
 \hypertarget{index_usage}{}\section{Usage}\label{index_usage}
Have a look \hyperlink{fundamental_usage}{Fundamental Usage} for an explenation of the main k\+Buffer functions (with some examples)~\newline
 If you want to take the mean of your buffer, have a look at \hyperlink{mean}{Mean of buffer} \hypertarget{index_example}{}\section{Example code}\label{index_example}
An example code project is available under ../../test/x86. It isn\textquotesingle{}t well documented, but you can compile it for your system. 