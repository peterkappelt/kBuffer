\hypertarget{fundamental_usage_datatype}{}\section{Buffer datatype definition}\label{fundamental_usage_datatype}
A ringbuffer consists of variables, which can be accessed in a continuous way.~\newline
 You have to define, which datatype you want to have the elements.~\newline
 By default, the elements are unsigned 16bit integers (uint16\+\_\+t).~\newline
 You can change the datatype by defining it. This definition must be {\bfseries before} the inclusion of th \hyperlink{k_buffer_8h}{k\+Buffer.\+h} header file 
\begin{DoxyCode}
\textcolor{preprocessor}{#define bufferDatatype your\_datatype}
\end{DoxyCode}
 Instead of uint16\+\_\+t, you can insert (almost) any datatype you want. \hypertarget{fundamental_usage_init}{}\section{Initializing a ringbuffer}\label{fundamental_usage_init}
At first, you have to include the k\+Buffer library into your project. This can be done by copying the files from src/k\+Buffer to your project\textquotesingle{}s directory. You can include the header as usual\+: 
\begin{DoxyCode}
\textcolor{preprocessor}{#include "\hyperlink{k_buffer_8h}{kBuffer.h}"}
\end{DoxyCode}
 In your code, you have to define an instance of \hyperlink{structbuffer__t}{buffer\+\_\+t}. You have to init this instance with the function \hyperlink{k_buffer_8c_aec18d6ea571b1326dbeb7ca15f4969c0}{buffer\+Init()}. If you want to have a ringbuffer with 8 elements\+: 
\begin{DoxyCode}
\hyperlink{structbuffer__t}{buffer\_t} ringbuffer;
\hyperlink{k_buffer_8c_aec18d6ea571b1326dbeb7ca15f4969c0}{bufferInit}(&ringbuffer, 8);
\end{DoxyCode}
 To check, if the initialization was successfull, you need to parse the return value of \hyperlink{k_buffer_8c_aec18d6ea571b1326dbeb7ca15f4969c0}{buffer\+Init()}\+: 
\begin{DoxyCode}
\hyperlink{structbuffer__t}{buffer\_t} ringbuffer;
\textcolor{keywordflow}{if}(\hyperlink{k_buffer_8c_aec18d6ea571b1326dbeb7ca15f4969c0}{bufferInit}(&ringbuffer, 8) == \hyperlink{k_buffer_8h_a7a0bf550b7bd49d85172e409c0034fe6a69e32851bd2f089b06555decd80aac1b}{bufferOK})\{
 do\_something\_it\_worked\_ok();
\}\textcolor{keywordflow}{else}\{
 do\_something\_there\_was\_an\_error();
\}
\end{DoxyCode}
 If you want to avoid the memory overhead of the dynamic memory allocation of the malloc() function you could use the \hyperlink{k_buffer_8c_a8da8904eb3cb9b87699cf0f45ce5bf51}{buffer\+Init\+Static()} function. 
\begin{DoxyCode}
\hyperlink{structbuffer__t}{buffer\_t} ringbuffer;
\hyperlink{k_buffer_8h_ae8d6ebfbda34ebc2e00138c04b46e9b1}{bufferDatatype} ringbufferPayload[8];
\hyperlink{k_buffer_8c_a8da8904eb3cb9b87699cf0f45ce5bf51}{bufferInitStatic}(&ringbuffer, 8, &ringbufferPayload[0]);
\end{DoxyCode}
 \hypertarget{fundamental_usage_write}{}\section{Writing data to the buffer}\label{fundamental_usage_write}
To write data to the buffer, you can use the \hyperlink{k_buffer_8c_a9d6410a89adf65a3ef12340ecb9bbd55}{buffer\+Write()} function\+: 
\begin{DoxyCode}
\textcolor{preprocessor}{#include "\hyperlink{k_buffer_8h}{kBuffer.h}"}

\textcolor{keywordtype}{int} main(\textcolor{keywordtype}{void})\{

 \hyperlink{structbuffer__t}{buffer\_t} ringbuffer;            \textcolor{comment}{// Declare an buffer instance}
 \hyperlink{k_buffer_8c_aec18d6ea571b1326dbeb7ca15f4969c0}{bufferInit}(&ringbuffer, 8);     \textcolor{comment}{// Init the buffer with 8 elements}
 \textcolor{comment}{//Notice, that no errorhandling has been done. We just expect a success}
 
 \hyperlink{k_buffer_8c_a9d6410a89adf65a3ef12340ecb9bbd55}{bufferWrite}(&ringbuffer, 42);   \textcolor{comment}{// Write the integer "42" to the buffer.}

 \textcolor{keywordflow}{return} 0;
\}
\end{DoxyCode}
 \hypertarget{fundamental_usage_read}{}\section{Reading data from the buffer}\label{fundamental_usage_read}
To read data from the buffer, you can use the \hyperlink{k_buffer_8c_a9b80be9033ccd6b5a101f811520ab4cc}{buffer\+Read()} function\+: 
\begin{DoxyCode}
\textcolor{preprocessor}{#include "\hyperlink{k_buffer_8h}{kBuffer.h}"}

\textcolor{keywordtype}{int} main(\textcolor{keywordtype}{void})\{

 \hyperlink{structbuffer__t}{buffer\_t} ringbuffer;                \textcolor{comment}{// Declare an buffer instance}
 \hyperlink{k_buffer_8c_aec18d6ea571b1326dbeb7ca15f4969c0}{bufferInit}(&ringbuffer, 8);         \textcolor{comment}{// Init the buffer with 8 elements}
 \textcolor{comment}{//Notice, that no errorhandling has been done. We just expect a success}
 
 \hyperlink{k_buffer_8c_a9d6410a89adf65a3ef12340ecb9bbd55}{bufferWrite}(&ringbuffer, 42);       \textcolor{comment}{// Write the integer "42" to the buffer.}

 uint16\_t dataRead;                  \textcolor{comment}{// Declare an integer, where the read data should be stored}
 \hyperlink{k_buffer_8c_a9b80be9033ccd6b5a101f811520ab4cc}{bufferRead}(&ringbuffer, &dataRead); \textcolor{comment}{// We expect, that dataRead is now 42 (because we have
       written 42 to the buffer before)}

 \textcolor{keywordflow}{return} 0;
\}
\end{DoxyCode}
 